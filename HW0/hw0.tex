\documentclass[11pt]{amsart}
\usepackage{geometry}                % See geometry.pdf to learn the layout options. There are lots.
\geometry{letterpaper}                   % ... or a4paper or a5paper or ... 
%\geometry{landscape}                % Activate for for rotated page geometry
%\usepackage[parfill]{parskip}    % Activate to begin paragraphs with an empty line rather than an indent
\usepackage{graphicx}
\usepackage{amssymb}
\usepackage[all]{xy}
\usepackage{epstopdf}
\usepackage{hyperref}

\DeclareGraphicsRule{.tif}{png}{.png}{`convert #1 `dirname #1`/`basename #1 .tif`.png}


% these packages make it easy to include figures in the text. 
\usepackage{float}
\restylefloat{figure}

\newcommand{\cX}{\mathcal{X}}
\newcommand{\cC}{\mathcal{C}}
\newcommand{\cF}{\mathcal{F}}




\begin{document}
{\Large Name: Eyasu Nigussie}  \\
\begin{center}
\Large AMATH 515 \hskip 2in Homework Set 0\\
{\bf Due:  Wednesday, January 10th, by 11 pm}. 
\end{center}

\bigskip


\section{Theory}
\begin{enumerate}
	\item Submit your write-up to Gradescope. Look for the assignment "Homework 0 -- theory".

\item {\bf Calculus primer}. For a function $f:\mathbb{R}^n \rightarrow \mathbb{R}$, we define the 
{\it gradient} to be the vector of partial derivatives: 
\[
\nabla f(x) = \begin{bmatrix} \frac{\partial f}{\partial x_1} \\ \vdots \\ \frac{\partial f}{\partial x_n}\end{bmatrix}
\]
and the {\it Hessian} to be the matrix of second partial derivatives: 
\[
\nabla^2 f(x) = 
\begin{bmatrix} \frac{\partial^2 f}{\partial x_1 \partial x_1}  & \dots & \frac{\partial^2 f}{\partial x_1 \partial x_n}
\\ \vdots 
\\ \frac{\partial^2 f}{\partial x_n \partial x_1} & \dots & \frac{\partial^2 f}{\partial x_n \partial x_n}\end{bmatrix}
\]
Compute the gradients and hessians of the following functions, with $x \in \mathbb{R}^4$ in all three examples. 
\begin{enumerate}
\item $f(x) = \sin(x_1 + x_2 + x_3 + x_4)$ \\ 
\[
\nabla f(x)= 
 \begin{bmatrix} \cos(x_1+x_2+x_3+x_4) \\ \cos(x_1+x_2+x_3+x_4) \\ \cos(x_1+x_2+x_3+x_4) \\ \cos(x_1+x_2+x_3+x_4)\end{bmatrix}
\] 
\[
\nabla^2 f(x) =
- \begin{bmatrix} \sin(x)\, \sin(x)\, \sin(x)\, \sin(x) \\
                  \sin(x)\, \sin(x)\, \sin(x)\, \sin(x) \\
                  \sin(x)\, \sin(x)\, \sin(x)\, \sin(x) \\
                  \sin(x)\, \sin(x)\, \sin(x)\, \sin(x) \\               
    \end{bmatrix}
\]

\item $f(x) = \|x\|^2 = x_1^2 + x_2^2 + x_3^2 + x_4^2$\\
\[
\nabla f(x) = 2
\begin{bmatrix}
    x_1 \\ x_2 \\ x_3 \\ x_4
\end{bmatrix}
\]
\[
\nabla^2 f(x) = 2
\begin{bmatrix}
    1\quad 0\quad 0\quad 0 \\
    0\quad 1\quad 0\quad 0 \\
    0\quad 0\quad 1\quad 0 \\
    0\quad 0\quad 0\quad 1 \\
\end{bmatrix}
=2I_4
\] 

\item $f(x) = \ln(x_1x_2x_3x_4)$.  
\[
\nabla f(x) = 
\begin{bmatrix}
    \frac{1}{x_1} \\
    \frac{1}{x_2} \\
    \frac{1}{x_3} \\
    \frac{1}{x_4} 
\end{bmatrix}
\]
\[
\nabla^2 f(x) = -
\begin{bmatrix}
    \frac{1}{x_1^2}\quad 0\quad 0\quad 0 \\
    0\quad \frac{1}{x_2^2}\quad 0\quad 0 \\
    0\quad 0\quad \frac{1}{x_3^2}\quad 0 \\ 
    0\quad 0\quad 0\quad \frac{1}{x_4^2} \\
\end{bmatrix}
\]
\end{enumerate}

\newpage

\item {\bf Linear algebra primer.}
\begin{enumerate}
\item What are the eigenvalues of the following matrix: 
\[
\begin{bmatrix}
1 & 0 & 0 & 0 \\
\pi & 2 & 0 & 0 \\
64 & -15 & 3 & 0 \\ 
321 & 0 & 0 & 5 
\end{bmatrix}
\]
$\lambda = \{1,2,3,5\}$ \\
\item Write down bases for the range and nullspace of the following matrix, written as the outer product of two vectors: 
\[
A = \begin{bmatrix}
1 \\ 0 \\ 1
\end{bmatrix} 
\begin{bmatrix}
1 & 1 & 1
\end{bmatrix} 
\]
Base for the range = $\left\{ \begin{bmatrix} 1 \\ 0 \\ 1\end{bmatrix}\right\}$ \\
\bigbreak
Basis for nullspace = $\left\{ \begin{bmatrix} -1 \\ 1 \\ 0\end{bmatrix}, \begin{bmatrix} -1 \\0 \\1 \end{bmatrix}\right\}$
\item Let $A$ be a $10 \times 5$ matrix, and $b$ a vector in $\mathbb{R}^{10}$.  The notation $A^T$ denotes the {\it transpose} of $A$, where the columns of $A$ are rows of $A^T$. 
\begin{itemize}
\item What is the size of $A^TA$? What is the size of $A^Tb$? \\
The size of A and B is $5 \times 5$ and $5 \times 1$ respectively. 
\item How many solutions might there be to the system $Ax = b$? \\
Since the system is over-determined, it could have no solution or a unique solution depending on the linear combination of the equations.
\item How many solutions might there be to the system $A^TA x = A^T b$? \\
It will have a unique solution if $A^TA$ is invertible. If not, it could have no solution or infinitely many solutions. 
\item Suppose the columns of $A$ are linearly independent. How many solutions might there be to the system $Ax=b$? To the system $A^TAx = A^Tb$? \\
$Ax=b$ will have a unique solution if $b \in col A.$ The system $A^TAx = A^Tb$ will have a unique solution.
\end{itemize}




\end{enumerate} 



\end{enumerate}

\bigskip \bigskip


\end{document}  

